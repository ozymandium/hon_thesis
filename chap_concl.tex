%%%%%%%%%%%%%%%%%%%%%%%%%%%%%%%%%%%%%%%%%%%%%%%%%%%%%%%%%%%%%%%%%%%%%%%%%%%%%%%%
%   Chapter 5
%%%%%%%%%%%%%%%%%%%%%%%%%%%%%%%%%%%%%%%%%%%%%%%%%%%%%%%%%%%%%%%%%%%%%%%%%%%%%%%%
\chapter{Conclusions \& Future Work} \label{chap:concl}

Two graphical tools were developed to assist drivers in convoys with high fidelity to the lead vehicle's path while maintaining a safe spacing between them. 
Experimentation reveals that they were quite effective in helping to enforce safe curvilinear following distances, while more development is needed to optimize results for lateral path deviation.

The zero landmark test was by far the most successful in garnering qualitative feedback. Drivers suggested a projected path overlay which gave a prediction of the follower's path in realtime given course and yaw rate information. This would counter the effects of being unable to tell how present actions would effect deviation in the approximately 1 s required for steering input to be reflected in either GUI. Furthermore, it was suggested, a model prediction scheme as in \cite{williamthesis} could be employed to show not only predicted path over some window, but estimate the current position and overlay it. Future work will pursue adding these improvements to the monolithic GUI and continuing to refine that tool, as it has been shown more successful than the Earth GUI at accomplished the goals outlined herein.